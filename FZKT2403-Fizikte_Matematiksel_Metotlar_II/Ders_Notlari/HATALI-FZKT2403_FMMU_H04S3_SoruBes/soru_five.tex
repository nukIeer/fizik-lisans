\documentclass[11pt,letterpaper,twocolumn]{fenbil}
\usepackage{amsmath}
\usepackage{amssymb}
\usepackage{graphicx}
\usepackage{tikz}
\usepackage[utf8]{inputenc}
\usepackage[turkish]{babel}

\title{FMM - Fizikte Matematiksel Metotlar - Uygulama Devam}
\author{Celal Ekrem Torun}
\date{10 Mart 2025}

\begin{document}
\twocolumn[\{@twocolumnfalse}

\begin{minipage}{0.15\textwidth}{
    }
\end{minipage}
\hspace{25pt}
\begin{minipage}{0.75\textwidth}
\vspace{5mm}
\Large{\textbf{FMM - Fizikte Matematiksel Metotlar - Uygulama Devam}}
    \vspace{3mm}

    \large{\textbf{Hazırlayan}; Celal Ekrem Torun}
    \vspace{2mm}

    \fontsize{0.35cm}{0.5cm}\selectfont \textit{Fizik Bölümü, İstanbul Üniversitesi\newline
    Beyazıt, Fatih, İstanbul, Türkiye}

\end{minipage}

\small

\@twocolumnfalse}]

\section{Uygulama}

\subsection*{Soru}
$S_1: x^2 + y^2 + z^2 = 9$ ve $S_2: z = x^2 + y^2 - 3$ yüzeyleri veriliyor. Bu yüzeylerin $P(2, -1, 2)$ noktasındaki teğet düzlemleri arasındaki açıyı bulunuz.

\textbf{Çözüm:}
İki yüzey arasındaki açı, yüzeylerin normal vektörleri arasındaki açıdır.

Adım 1: Her yüzey için gradyan vektörlerini hesaplayın:
$S_1: f(x, y, z) = x^2 + y^2 + z^2 - 9 = 0$
$S_2: g(x, y, z) = x^2 + y^2 - z - 3 = 0$

Gradyanları hesaplayalım:
\[
\nabla f = \frac{\partial f}{\partial x}\hat{i} + \frac{\partial f}{\partial y}\hat{j} + \frac{\partial f}{\partial z}\hat{k} = (2x)\hat{i} + (2y)\hat{j} + (2z)\hat{k}
\]
\[
\nabla g = \frac{\partial g}{\partial x}\hat{i} + \frac{\partial g}{\partial y}\hat{j} + \frac{\partial g}{\partial z}\hat{k} = (2x)\hat{i} + (2y)\hat{j} + (-1)\hat{k}
\]

Adım 2: Gradyanları $P(2, -1, 2)$ noktasında değerlendirin:
\[
\nabla f(2, -1, 2) = (2(2))\hat{i} + (2(-1))\hat{j} + (2(2))\hat{k} = 4\hat{i} - 2\hat{j} + 4\hat{k}
\]
\[
\nabla g(2, -1, 2) = (2(2))\hat{i} + (2(-1))\hat{j} + (-1)\hat{k} = 4\hat{i} - 2\hat{j} - \hat{k}
\]

Adım 3: Gradyanların iç çarpımını ve büyüklüklerini hesaplayın:
\[
\nabla f \cdot \nabla g = (4)(4) + (-2)(-2) + (4)(-1) = 16 + 4 - 4 = 16
\]
\[
|\nabla f| = \sqrt{4^2 + (-2)^2 + 4^2} = \sqrt{16 + 4 + 16} = \sqrt{36} = 6
\]
\[
|\nabla g| = \sqrt{4^2 + (-2)^2 + (-1)^2} = \sqrt{16 + 4 + 1} = \sqrt{21}
\]

Adım 4: Açı kosinüsünü hesaplayın:
\[
\cos{\theta} = \frac{|\nabla f \cdot \nabla g|}{|\nabla f| |\nabla g|} = \frac{|16|}{6 \sqrt{21}} = \frac{8}{3\sqrt{21}}
\]

Adım 5: Açıyı bulun:
\[
\theta = \arccos{\left(\frac{8}{3\sqrt{21}}\right)}
\]
\textbf{Sonuç olarak}, yüzeyler arasındaki açı $\arccos{\left(\frac{8}{3\sqrt{21}}\right)}$'dir.

\end{document}
