\documentclass[11pt,letterpaper,twocolumn]{fenbil}
\usepackage[utf8]{inputenc}
\usepackage[T1]{fontenc}
\usepackage{amsmath}
\usepackage{amsfonts}
\usepackage{amssymb}
\usepackage{physics}
\usepackage{bm}
\usepackage{graphicx}
\usepackage{geometry}
\usepackage{mathtools}

\geometry{margin=2.5cm}

\begin{document}

\twocolumn[
\begin{@twocolumnfalse}
\begin{minipage}{0.15\textwidth}
{
\includegraphics[width=4cm]{logo/iufizik.png}
}
\end{minipage}
\hspace{25pt}
\begin{minipage}{0.75\textwidth}
\vspace{5mm}
\Large{\textbf{FİZİKTE MATEMATİKSEL METOTLAR \\ 13 MART 2025 Eğrisel Koordinatlar ve Diferansiyel Elemanlar}}
\vspace{3mm}\\
\large{\textbf{Ad Soyad:} Celal Ekrem Torun - 0411230037}
\vspace{2mm}\\
\large{\textbf{DERS:} Prof. Dr. ERTAN GUDEKLI}\newline
\fontsize{0.35cm}{0.5cm}\selectfont
\textit{Fizik Bölümü, İstanbul Üniversitesi\newline
Beyazıt, Fatih, İstanbul, Türkiye\newline
13 Mart 2025}
\end{minipage}
\small
\end{@twocolumnfalse}]

\hspace{25pt}
\hspace{25pt}
\hspace{25pt}

\section{Yay Uzunluğu Elemanı ($ds^2$)}

\subsection{Ortogonal (Dik) Koordinat Sistemlerinde Yay Uzunluğu}

Bir koordinat sistemi ortogonal (dik) ise, yay uzunluğu elemanı şu şekilde yazılır:

\begin{equation}
ds^2 = g_{11}(du^1)^2 + g_{22}(du^2)^2 + g_{33}(du^3)^2
\end{equation}

Bu ifadede:
\begin{itemize}
\item $g_{11}$, $g_{22}$, $g_{33}$: Metrik tensörün diyagonal elemanları
\item $du^1$, $du^2$, $du^3$: Koordinat diferansiyelleri
\end{itemize}

\subsection{Yay Uzunluğunun Genel İfadesi}

Yay uzunluğu elemanı genel olarak şu şekilde yazılır:

\begin{equation}
ds^2 = g_{ij}du^i du^j
\end{equation}

Bu ifade, Einstein toplam gösterimi kullanılarak yazılmıştır ve tekrarlanan indisler üzerinden toplam alınacağını gösterir. Açık şekilde yazarsak:

\begin{equation}
ds^2 = \sum_{i=1}^3 \sum_{j=1}^3 g_{ij}du^i du^j
\end{equation}

\subsection{Tanjant Vektörleri ve Metrik Bağıntısı}

Tanjant vektörleri ($\vec{T}_i$) kullanarak yay uzunluğunu şöyle ifade ederiz:

\begin{equation}
ds^2 = \vec{T}_i \cdot \vec{T}_j \, du^i du^j
\end{equation}

Burada $\vec{T}_i = \frac{\partial \vec{r}}{\partial u^i}$ ifadesi, konum vektörünün ($\vec{r}$) koordinatlara göre kısmi türevlerini temsil eder.

Ayrıca, tanjant vektörleri şu şekilde de yazılabilir:

\begin{equation}
\vec{T}_i = h_i \vec{e}_i
\end{equation}

Burada:
\begin{itemize}
\item $h_i$: Ölçek faktörü (veya Lamé katsayısı)
\item $\vec{e}_i$: Lokal birim vektör
\end{itemize}

Bu durumda yay uzunluğu elemanı:

\begin{equation}
ds^2 = h_i \vec{e}_i \cdot h_j \vec{e}_j \, du^i du^j = h_i h_j \vec{e}_i \cdot \vec{e}_j \, du^i du^j
\end{equation}

Ortogonal sistemlerde, $\vec{e}_i \cdot \vec{e}_j = \delta_{ij}$ (Kronecker delta) olduğundan:

\begin{equation}
ds^2 = h_i^2 (du^i)^2 \quad \text{(i için toplam gösterimi)}
\end{equation}

Bu, şu anlama gelir:
\begin{itemize}
\item Eğer $i = j$ ise, $\vec{e}_i \cdot \vec{e}_j = 1$
\item Eğer $i \neq j$ ise, $\vec{e}_i \cdot \vec{e}_j = 0$
\end{itemize}

\subsection{Metrik Tensör ve Ölçek Faktörleri Arasındaki İlişki}

Metrik tensör elemanları ile ölçek faktörleri arasında şu ilişkiler vardır:

\begin{enumerate}
\item Diyagonal elemanlar (i = j):
   \begin{equation}
   g_{ii} = h_i \cdot h_i = h_i^2
   \end{equation}

\item Diyagonal olmayan elemanlar (i $\neq$ j) (ortogonal sistemlerde):
   \begin{equation}
   g_{ij} = h_i h_j \cdot 0 = 0
   \end{equation}

\item Ölçek faktörleri:
   \begin{equation}
   h_i = \pm \sqrt{g_{ii}}
   \end{equation}
   
   Not: Fiziksel nedenlerden dolayı genellikle pozitif değer alınır.
\end{enumerate}

\section{Silindirik Koordinatlarda Metrik}

Silindirik koordinat sistemi $(\rho, \theta, z)$ için metrik elemanları:

\begin{equation}
g_{\rho\rho} = h_\rho^2 = 1
\end{equation}

\begin{equation}
g_{\theta\theta} = h_\theta^2 = \rho^2
\end{equation}

\begin{equation}
g_{zz} = h_z^2 = 1
\end{equation}

Diğer tüm $g_{ij} = 0$ (i $\neq$ j için, çünkü sistem ortogonaldir)

Ölçek faktörleri:
\begin{itemize}
\item $h_\rho = 1$
\item $h_\theta = \rho$
\item $h_z = 1$
\end{itemize}

Cartesian koordinatlardan silindirik koordinatlara dönüşüm:
\begin{itemize}
\item $x = \rho \cos \theta$
\item $y = \rho \sin \theta$
\item $z = z$
\end{itemize}

\section{Metrik Tensörü Doğrudan Hesaplama}

Metrik tensör elemanları, konum vektörünün koordinatlara göre türevleri kullanılarak doğrudan hesaplanabilir:

\begin{equation}
g_{kl} = \frac{\partial \vec{r}}{\partial u^k} \cdot \frac{\partial \vec{r}}{\partial u^l}
\end{equation}

Konum vektörü Cartesian koordinatlarda:

\begin{equation}
\vec{r} = x^i \vec{e}_i = x\vec{e}_x + y\vec{e}_y + z\vec{e}_z
\end{equation}

Bu durumda metrik tensör:

\begin{equation}
g_{kl} = \frac{\partial x^i}{\partial u^k} \frac{\partial x^j}{\partial u^l} \delta_{ij} = \sum_{i=1}^3 \frac{\partial x^i}{\partial u^k} \frac{\partial x^i}{\partial u^l}
\end{equation}

Daha açık bir gösterimle:

\begin{equation}
g_{kl} = \frac{\partial x}{\partial u^k} \frac{\partial x}{\partial u^l} + \frac{\partial y}{\partial u^k} \frac{\partial y}{\partial u^l} + \frac{\partial z}{\partial u^k} \frac{\partial z}{\partial u^l}
\end{equation}

\section{Silindirik Koordinatlarda Metrik Tensörün Hesaplanması}

Silindirik koordinatlarda metrik tensörü hesaplayalım:

\begin{enumerate}
\item Kısmi türevleri bulalım:
   \begin{align}
   \frac{\partial x}{\partial \rho} &= \cos \theta\\
   \frac{\partial x}{\partial \theta} &= -\rho \sin \theta\\
   \frac{\partial x}{\partial z} &= 0\\
   \frac{\partial y}{\partial \rho} &= \sin \theta\\
   \frac{\partial y}{\partial \theta} &= \rho \cos \theta\\
   \frac{\partial y}{\partial z} &= 0\\
   \frac{\partial z}{\partial \rho} &= 0\\
   \frac{\partial z}{\partial \theta} &= 0\\
   \frac{\partial z}{\partial z} &= 1
   \end{align}

\item Metrik elemanlarını hesaplayalım:
   
   \begin{align}
   g_{\rho\rho} &= \frac{\partial x}{\partial \rho}\frac{\partial x}{\partial \rho} + \frac{\partial y}{\partial \rho}\frac{\partial y}{\partial \rho} + \frac{\partial z}{\partial \rho}\frac{\partial z}{\partial \rho}\\
   &= \cos^2 \theta + \sin^2 \theta + 0 = 1
   \end{align}
   
   \begin{align}
   g_{\theta\theta} &= \frac{\partial x}{\partial \theta}\frac{\partial x}{\partial \theta} + \frac{\partial y}{\partial \theta}\frac{\partial y}{\partial \theta} + \frac{\partial z}{\partial \theta}\frac{\partial z}{\partial \theta}\\
   &= (-\rho \sin \theta)^2 + (\rho \cos \theta)^2 + 0 = \rho^2 \sin^2 \theta + \rho^2 \cos^2 \theta = \rho^2
   \end{align}
   
   \begin{align}
   g_{zz} &= \frac{\partial x}{\partial z}\frac{\partial x}{\partial z} + \frac{\partial y}{\partial z}\frac{\partial y}{\partial z} + \frac{\partial z}{\partial z}\frac{\partial z}{\partial z}\\
   &= 0 + 0 + 1 = 1
   \end{align}
   
   \begin{align}
   g_{\rho\theta} &= \frac{\partial x}{\partial \rho}\frac{\partial x}{\partial \theta} + \frac{\partial y}{\partial \rho}\frac{\partial y}{\partial \theta} + \frac{\partial z}{\partial \rho}\frac{\partial z}{\partial \theta}\\
   &= \cos \theta \cdot (-\rho \sin \theta) + \sin \theta \cdot (\rho \cos \theta) + 0\\
   &= -\rho \sin \theta \cos \theta + \rho \sin \theta \cos \theta = 0
   \end{align}
\end{enumerate}

Benzer şekilde, $g_{\rho z} = 0$ ve $g_{\theta z} = 0$ olduğunu gösterebiliriz.

\section{Hacim Elemanı (dV)}

\subsection{Genel Hacim Elemanı Formülü}

Üç boyutlu uzayda hacim elemanı, üç diferansiyel konum vektörünün karma çarpımının mutlak değeri olarak ifade edilir:

\begin{equation}
dV = |\vec{dr}_1 \cdot (\vec{dr}_2 \times \vec{dr}_3)|
\end{equation}

Burada $\vec{dr}_i = \vec{T}_i du^i$ (toplamı yok) vektörleridir.

Bu ifade, Jacobian determinantı kullanılarak şu şekilde yazılabilir:

\begin{equation}
dV = \left|\det\left(\frac{\partial(x,y,z)}{\partial(u^1,u^2,u^3)}\right)\right| du^1 du^2 du^3
\end{equation}

Açık şekilde yazarsak:

\begin{equation}
dV = \left|\begin{vmatrix} 
\frac{\partial x}{\partial u^1} & \frac{\partial x}{\partial u^2} & \frac{\partial x}{\partial u^3} \\
\frac{\partial y}{\partial u^1} & \frac{\partial y}{\partial u^2} & \frac{\partial y}{\partial u^3} \\
\frac{\partial z}{\partial u^1} & \frac{\partial z}{\partial u^2} & \frac{\partial z}{\partial u^3}
\end{vmatrix}\right| du^1 du^2 du^3
\end{equation}

\subsection{Tanjant Vektörleriyle Hacim Elemanı}

Tanjant vektörleri kullanarak hacim elemanı şu şekilde yazılabilir:

\begin{equation}
dV = |\vec{T}_1 \cdot (\vec{T}_2 \times \vec{T}_3)| du^1 du^2 du^3
\end{equation}

Ölçek faktörleri kullanılarak:

\begin{align}
dV &= |h_1 \vec{e}_1 \cdot (h_2 \vec{e}_2 \times h_3 \vec{e}_3)| du^1 du^2 du^3\\
&= h_1 h_2 h_3 |\vec{e}_1 \cdot (\vec{e}_2 \times \vec{e}_3)| du^1 du^2 du^3
\end{align}

Ortogonal koordinat sistemlerinde $|\vec{e}_1 \cdot (\vec{e}_2 \times \vec{e}_3)| = 1$ olduğundan:

\begin{equation}
dV = h_1 h_2 h_3 \, du^1 du^2 du^3 = \sqrt{g} \, du^1 du^2 du^3
\end{equation}

Burada $g = \det(g_{ij}) = h_1^2 h_2^2 h_3^2$ metrik tensörün determinantıdır.

\subsection{Silindirik Koordinatlarda Hacim Elemanı}

Silindirik koordinatlarda $(\rho, \theta, z)$ hacim elemanı:

\begin{equation}
dV = h_\rho h_\theta h_z \, d\rho \, d\theta \, dz = 1 \cdot \rho \cdot 1 \, d\rho \, d\theta \, dz = \rho \, d\rho \, d\theta \, dz
\end{equation}

\section{Alan Elemanı (dS)}

\subsection{İki Boyutlu Alan Elemanı}

İki koordinatın ($u^1$, $u^2$) oluşturduğu yüzeydeki alan elemanı, tanjant vektörlerinin vektörel çarpımı olarak ifade edilir:

\begin{equation}
d\vec{S} = \frac{\partial \vec{F}}{\partial u^1} \times \frac{\partial \vec{F}}{\partial u^2} \, du^1 du^2
\end{equation}

Alan elemanının büyüklüğü:

\begin{equation}
dS = \left|\frac{\partial \vec{F}}{\partial u^1} \times \frac{\partial \vec{F}}{\partial u^2}\right| \, du^1 du^2
\end{equation}

Bu, determinant formunda yazılabilir:

\begin{equation}
dS = \left|\begin{vmatrix}
\frac{\partial x}{\partial u^1} & \frac{\partial x}{\partial u^2} \\
\frac{\partial y}{\partial u^1} & \frac{\partial y}{\partial u^2}
\end{vmatrix}\right| \, du^1 du^2
\end{equation}

Genel olarak, ortogonal koordinatlarda:

\begin{equation}
dS = h_i h_j \, du^i du^j \quad \text{(i ve j sabit, k hariç)}
\end{equation}

Burada $i$, $j$ ve $k$ birbirinden farklı indislerdir.

\subsection{Silindirik Koordinatlarda Alan Elemanı}

Silindirik koordinatlarda $(\rho, \theta)$ düzleminde alan elemanı:

\begin{align}
dS_{\rho\theta} &= \left|\begin{vmatrix}
\frac{\partial x}{\partial \rho} & \frac{\partial x}{\partial \theta} \\
\frac{\partial y}{\partial \rho} & \frac{\partial y}{\partial \theta}
\end{vmatrix}\right| \, d\rho \, d\theta\\
&= \left|\begin{vmatrix}
\cos \theta & -\rho \sin \theta \\
\sin \theta & \rho \cos \theta
\end{vmatrix}\right| \, d\rho \, d\theta
\end{align}

\begin{align}
&= |\rho \cos^2 \theta + \rho \sin^2 \theta| \, d\rho \, d\theta = \rho \, d\rho \, d\theta
\end{align}

\section{Uygulama Örneği}

Bir integral hesaplama örneği:

\begin{equation}
A = \iint \sqrt{x^2 + y^2} \, dx \, dy
\end{equation}

Bu integrali kutupsal koordinatlarda hesaplayalım:
\begin{itemize}
\item $x = \rho \cos \theta$
\item $y = \rho \sin \theta$
\item $\sqrt{x^2 + y^2} = \rho$
\item $dx \, dy = \rho \, d\rho \, d\theta$ (Jacobian)
\end{itemize}

İntegral dönüşümü:

\begin{equation}
A = \iint \rho \cdot \rho \, d\rho \, d\theta = \iint \rho^2 \, d\rho \, d\theta
\end{equation}

İntegrasyon sınırları belirlenerek:

\begin{align}
A &= \int_0^{2\pi} \int_0^r \rho^2 \, d\rho \, d\theta = \int_0^{2\pi} d\theta \int_0^r \rho^2 \, d\rho\\
&= 2\pi \cdot \left[\frac{\rho^3}{3}\right]_0^r = 2\pi \cdot \frac{r^3}{3} = \frac{2\pi r^3}{3}
\end{align}

Bu örnekte, Cartesian koordinatlardan kutupsal koordinatlara geçiş yaparak hesaplama kolaylaştırılmıştır.

\end{document}