\documentclass[11pt,letterpaper,twocolumn]{fenbil}

\usepackage{amsmath}
\usepackage{amssymb}
\usepackage{graphicx}
\usepackage{tikz}
\usepackage[utf8]{inputenc}
\usepackage[turkish]{babel}

\title{FMM - Fizikte Matematiksel Metotlar - Uygulama Devam}
\author{Celal Ekrem Torun}
\date{10 Mart 2025}

\begin{document}

\begin{minipage}{0.15\textwidth}
{}
\end{minipage}
\hspace{25pt}
\begin{minipage}{0.75\textwidth}
\vspace{5mm}
\Large{\textbf{FMM - Fizikte Matematiksel Metotlar - Uygulama Devam}}
\vspace{3mm}
\large{\textbf{Hazırlayan}; Celal Ekrem Torun}
\vspace{2mm}
\fontsize{0.35cm}{0.5cm}\selectfont \textit{Fizik Bölümü, İstanbul Üniversitesi\newline
Beyazıt, Fatih, İstanbul, Türkiye}
\end{minipage}
\small

\vspace{2mm}
\vspace{2mm}
\vspace{2mm}
\section{Uygulama}

\subsection*{Soru 5}

$S_1: x^2 + y^2 + z^2 = 9$ ve $S_2: z = x^2 + y^2 - 3$ yüzeyleri veriliyor. Bu yüzeylerin $P(2, -1, 2)$ noktasındaki teğet düzlemleri arasındaki açıyı bulunuz.

\textbf{Çözüm:}

İki yüzey arasındaki açı, yüzeylerin normal vektörleri arasındaki açıdır.

Adım 1: Her yüzey için gradyan vektörlerini hesaplayalım:

$S_1: f(x, y, z) = x^2 + y^2 + z^2 - 9 = 0$

$S_2: g(x, y, z) = z - x^2 - y^2 + 3 = 0$

Gradyanları hesaplayalım:

\begin{align}
\vec{\nabla} f &= \frac{\partial f}{\partial x}\hat{\mathbf{i}} + \frac{\partial f}{\partial y}\hat{\mathbf{j}} + \frac{\partial f}{\partial z}\hat{\mathbf{k}} \\
&= 2x\hat{\mathbf{i}} + 2y\hat{\mathbf{j}} + 2z\hat{\mathbf{k}}
\end{align}

\begin{align}
\vec{\nabla} g &= \frac{\partial g}{\partial x}\hat{\mathbf{i}} + \frac{\partial g}{\partial y}\hat{\mathbf{j}} + \frac{\partial g}{\partial z}\hat{\mathbf{k}} \\
&= -2x\hat{\mathbf{i}} - 2y\hat{\mathbf{j}} + 1\hat{\mathbf{k}}
\end{align}

Adım 2: Gradyanları $P(2, -1, 2)$ noktasında değerlendirelim:

\begin{align}
\vec{\nabla} f(2, -1, 2) &= 2(2)\hat{\mathbf{i}} + 2(-1)\hat{\mathbf{j}} + 2(2)\hat{\mathbf{k}} \\
&= 4\hat{\mathbf{i}} - 2\hat{\mathbf{j}} + 4\hat{\mathbf{k}}
\end{align}

\begin{align}
\vec{\nabla} g(2, -1, 2) &= -2(2)\hat{\mathbf{i}} - 2(-1)\hat{\mathbf{j}} + 1\hat{\mathbf{k}} \\
&= -4\hat{\mathbf{i}} + 2\hat{\mathbf{j}} + 1\hat{\mathbf{k}}
\end{align}

Adım 3: Normal vektörler arasındaki açıyı hesaplayalım. İki vektör arasındaki açı şu formülle bulunur:

\begin{align}
\cos\theta = \frac{\vec{\nabla} f \cdot \vec{\nabla} g}{|\vec{\nabla} f||\vec{\nabla} g|}
\end{align}

Önce iç çarpımı hesaplayalım:
\begin{align}
\vec{\nabla} f \cdot \vec{\nabla} g &= (4)(-4) + (-2)(2) + (4)(1) \\
&= -16 - 4 + 4 \\
&= -16
\end{align}

\vspace{2mm}
\vspace{2mm}
\vspace{2mm}
\vspace{2mm}
\hspace{10mm}
Vektörlerin büyüklüklerini hesaplayalım:
\begin{align}
|\vec{\nabla} f| &= \sqrt{4^2 + (-2)^2 + 4^2} \\
&= \sqrt{16 + 4 + 16} \\
&= \sqrt{36} \\
&= 6
\end{align}

\begin{align}
|\vec{\nabla} g| &= \sqrt{(-4)^2 + 2^2 + 1^2} \\
&= \sqrt{16 + 4 + 1} \\
&= \sqrt{21}
\end{align}

Adım 4: Açı kosinüsünü hesaplayalım:
\begin{align}
\cos\theta &= \frac{\vec{\nabla} f \cdot \vec{\nabla} g}{|\vec{\nabla} f||\vec{\nabla} g|} \\
&= \frac{-16}{6 \cdot \sqrt{21}} \\
&= -\frac{16}{6\sqrt{21}} \\
&= -\frac{8}{3\sqrt{21}}
\end{align}

Adım 5: Açıyı bulalım:
\begin{align}
\theta = \arccos\left(-\frac{8}{3\sqrt{21}}\right)
\end{align}

\textbf{Sonuç olarak}, yüzeyler arasındaki açı $\theta = \arccos\left(-\frac{8}{3\sqrt{21}}\right)$ radyandır.

Sayısal değer olarak hesaplarsak:
\begin{align}
\frac{8}{3\sqrt{21}} &\approx \frac{8}{3 \cdot 4.583} \\
&\approx \frac{8}{13.749} \\
&\approx 0.582
\end{align}

Dolayısıyla $\theta \approx \arccos(-0.582) \approx 125.5^{\circ}$ olarak bulunur.

\end{document}