\documentclass[11pt,letterpaper,twocolumn]{fenbil}
\usepackage[utf8]{inputenc}
\usepackage[T1]{fontenc}
\usepackage{amsmath}
\usepackage{amsfonts}
\usepackage{amssymb}
\usepackage{physics}
\usepackage{bm}
\usepackage{graphicx}
\usepackage{geometry}
\usepackage{mathtools}

\geometry{margin=2.5cm}

\begin{document}

\twocolumn[
\begin{@twocolumnfalse}
\begin{minipage}{0.15\textwidth}
{
\includegraphics[width=4cm]{logo/iufizik.png}
}
\end{minipage}
\hspace{25pt}
\begin{minipage}{0.75\textwidth}
\vspace{5mm}
\Large{\textbf{FİZİKTE MATEMATİKSEL METOTLAR \\ 6 MART 2025 }}
\vspace{3mm}\\
\large{\textbf{Ad Soyad:} Celal Ekrem Torun - 0411230037}
\vspace{2mm}\\
\large{\textbf{DERS:} Prof. Dr. ERTAN GUDEKLI}\newline
\fontsize{0.35cm}{0.5cm}\selectfont
\textit{Fizik Bölümü, İstanbul Üniversitesi\newline
Beyazıt, Fatih, İstanbul, Türkiye\newline
13 Mart 2025}
\end{minipage}
\small
\end{@twocolumnfalse}]

\hspace{25pt}
\hspace{25pt}
\hspace{25pt}


\section*{Temel Kavramlar}
\begin{itemize}
    \item Kısmi türevler genellikle $\partial$ sembolü ile gösterilir.
    \item Koordinat sistemlerinde birim vektörler, sistemin geometrisini tanımlamak için önemlidir.
    \item Silindirik koordinat sistemi $(\rho, \theta, z)$ ile tanımlanır.
    \item Küresel koordinat sistemi $(r, \theta, \phi)$ ile tanımlanır.
\end{itemize}

\section*{Soru 2: Silindirik Koordinat Sisteminin Dikliğinin İspatı}

\subsection*{Yöntem 1: Birim Vektörlerin Skaler Çarpımı}

Bir koordinat sisteminin dik olması için, sistemin birim vektörlerinin birbirine dik (ortogonal) olması gerekir. Silindirik koordinat sisteminde birim vektörler $\vec{e}_\rho$, $\vec{e}_\theta$ ve $\vec{e}_z$'dir.

İki vektörün birbirine dik olması için skaler çarpımlarının sıfır olması gerekir. Bunu şu şekilde kontrol edelim:

\begin{align}
    \vec{e}_\rho \cdot \vec{e}_\theta &= 0 \\
    \vec{e}_\rho \cdot \vec{e}_z &= 0 \\
    \vec{e}_\theta \cdot \vec{e}_z &= 0
\end{align}

Bu skaler çarpımların sıfır olduğunu gösterebiliriz:

1) $\vec{e}_\rho \cdot \vec{e}_\theta$: $\vec{e}_\rho$ $\rho$ yönündedir ve $\vec{e}_\theta$ teğetsel yöndedir. Geometrik olarak bunlar birbirine diktir, dolayısıyla çarpımları sıfırdır.

2) $\vec{e}_\rho \cdot \vec{e}_z$: $\vec{e}_\rho$ $xy$-düzlemindedir, $\vec{e}_z$ ise $z$-eksenindedir ve bu iki yön birbirine diktir, dolayısıyla çarpımları sıfırdır.

3) $\vec{e}_\theta \cdot \vec{e}_z$: $\vec{e}_\theta$ $xy$-düzlemindedir ve dairesel harekete teğettir, $\vec{e}_z$ ise $z$-eksenindedir. Bu ikisi birbirine diktir, dolayısıyla çarpımları sıfırdır.

\subsection*{Yöntem 2: Teğet Bileşenlerin Kontrolü}

Koordinat eğrilerine teğet olan birim vektörler için de benzer bir kontrol yapabiliriz:

\begin{align}
    \vec{t}_\rho \cdot \vec{t}_\theta &= 0 \\
    \vec{t}_\rho \cdot \vec{t}_z &= 0 \\
    \vec{t}_\theta \cdot \vec{t}_z &= 0
\end{align}

Burada $\vec{t}_\rho$, $\vec{t}_\theta$ ve $\vec{t}_z$, ilgili koordinat eğrilerine teğet vektörlerdir.

Silindirik koordinat sisteminde:
\begin{itemize}
    \item $\vec{t}_\rho$ radyal yönde (merkezden dışarıya doğru)
    \item $\vec{t}_\theta$ açısal yönde (dairesel hareket yönünde)
    \item $\vec{t}_z$ dikey yönde
\end{itemize}

Bu üç yön birbirine dik olduğundan, teğet vektörlerin skaler çarpımları sıfırdır. Dolayısıyla silindirik koordinat sistemi diktir.

\section*{Soru 3: Silindirik ve Kartezyen Koordinatlar Arasındaki Dönüşüm}

\subsection*{Adım 1: Birim Vektörlerin İfadesi}

Silindirik koordinatlardaki birim vektörleri Kartezyen birim vektörleri cinsinden ifade edelim:

\begin{align}
    \vec{e}_\rho &= \cos\theta \, \vec{e}_x + \sin\theta \, \vec{e}_y \\
    \vec{e}_\theta &= -\sin\theta \, \vec{e}_x + \cos\theta \, \vec{e}_y \\
    \vec{e}_z &= \vec{e}_z
\end{align}

Bu ifadeleri geometrik olarak doğrulayabiliriz:
\begin{itemize}
    \item $\vec{e}_\rho$: Merkezden dışarıya doğru olan birim vektör, $x$-ekseni ile $\theta$ açısı yapıyorsa, $x$ bileşeni $\cos\theta$, $y$ bileşeni $\sin\theta$ olur.
    \item $\vec{e}_\theta$: $\vec{e}_\rho$'ya dik olan teğet birim vektör, $x$ bileşeni $-\sin\theta$, $y$ bileşeni $\cos\theta$ olur.
    \item $\vec{e}_z$: Dikey eksen her iki sistemde de aynıdır.
\end{itemize}

\subsection*{Adım 2: Koordinat Dönüşüm İlişkileri}

Silindirik ve Kartezyen koordinatlar arasındaki dönüşüm ilişkileri:

\begin{align}
    x &= \rho \cos\theta \\
    y &= \rho \sin\theta \\
    z &= z
\end{align}

Tersi:
\begin{align}
    \rho &= \sqrt{x^2 + y^2} \\
    \theta &= \arctan\left(\frac{y}{x}\right) \\
    z &= z
\end{align}

\subsection*{Adım 3: Dönüşüm Matrisi}

Birim vektörler arasındaki ilişkiyi matris formunda yazalım:

\begin{align}
    \begin{bmatrix}
        \vec{e}_\rho \\
        \vec{e}_\theta \\
        \vec{e}_z
    \end{bmatrix} = 
    \begin{bmatrix}
        \cos\theta & \sin\theta & 0 \\
        -\sin\theta & \cos\theta & 0 \\
        0 & 0 & 1
    \end{bmatrix}
    \begin{bmatrix}
        \vec{e}_x \\
        \vec{e}_y \\
        \vec{e}_z
    \end{bmatrix}
\end{align}

Bu dönüşüm matrisi $A$ olsun. Matrisin özelliklerini inceleyelim:
\begin{itemize}
    \item $A$ ortogonal bir matristir: $A^T A = I$
    \item Ortogonal matris olduğundan $A^{-1} = A^T$
    \item Determinantı $|A| = 1$ olduğundan, bu dönüşüm uzunlukları ve açıları korur.
\end{itemize}

\subsection*{Adım 4: Matris Tersi ile Ters Dönüşüm}

Dönüşüm matrisinin tersi:
\begin{align}
    A^{-1} = A^T = 
    \begin{bmatrix}
        \cos\theta & -\sin\theta & 0 \\
        \sin\theta & \cos\theta & 0 \\
        0 & 0 & 1
    \end{bmatrix}
\end{align}

Dönüşüm matrislerini kullanarak şu ilişkileri yazabiliriz:
\begin{align}
    \vec{x} &= A\vec{y} \\
    \vec{y} &= A^{-1}\vec{x}
\end{align}
Burada $\vec{x}$ Kartezyen sistemi, $\vec{y}$ ise silindirik sistemi temsil eder.

\subsection*{Adım 5: Konum Vektörünün İfadesi}

Konum vektörü $\vec{r}$'yi her iki koordinat sisteminde de ifade edelim.

Kartezyen koordinatlarda:
\begin{align}
    \vec{r} &= x\vec{e}_x + y\vec{e}_y + z\vec{e}_z
\end{align}

Bu ifadeye silindirik koordinatları yerleştirelim:
\begin{align}
    \vec{r} &= \rho\cos\theta\vec{e}_x + \rho\sin\theta\vec{e}_y + z\vec{e}_z
\end{align}

Şimdi Kartezyen birim vektörlerini silindirik birim vektörleri cinsinden yazalım. Dönüşüm matrisinin tersi ile:
\begin{align}
    \begin{bmatrix}
        \vec{e}_x \\
        \vec{e}_y \\
        \vec{e}_z
    \end{bmatrix} = 
    \begin{bmatrix}
        \cos\theta & -\sin\theta & 0 \\
        \sin\theta & \cos\theta & 0 \\
        0 & 0 & 1
    \end{bmatrix}
    \begin{bmatrix}
        \vec{e}_\rho \\
        \vec{e}_\theta \\
        \vec{e}_z
    \end{bmatrix}
\end{align}

Yani:
\begin{align}
    \vec{e}_x &= \cos\theta\vec{e}_\rho - \sin\theta\vec{e}_\theta \\
    \vec{e}_y &= \sin\theta\vec{e}_\rho + \cos\theta\vec{e}_\theta \\
    \vec{e}_z &= \vec{e}_z
\end{align}

Bu ifadeleri konum vektöründe yerine koyalım:
\begin{align}
    \vec{r} &= \rho\cos\theta(\cos\theta\vec{e}_\rho - \sin\theta\vec{e}_\theta) + \rho\sin\theta(\sin\theta\vec{e}_\rho + \cos\theta\vec{e}_\theta) + z\vec{e}_z \\
    &= \rho\cos^2\theta\vec{e}_\rho - \rho\cos\theta\sin\theta\vec{e}_\theta + \rho\sin^2\theta\vec{e}_\rho + \rho\sin\theta\cos\theta\vec{e}_\theta + z\vec{e}_z
\end{align}

Terimlerini düzenleyelim:
\begin{align}
    \vec{r} &= \rho(\cos^2\theta + \sin^2\theta)\vec{e}_\rho + \rho(\sin\theta\cos\theta - \cos\theta\sin\theta)\vec{e}_\theta + z\vec{e}_z \\
    &= \rho \cdot 1 \cdot \vec{e}_\rho + \rho \cdot 0 \cdot \vec{e}_\theta + z\vec{e}_z \\
    &= \rho\vec{e}_\rho + z\vec{e}_z
\end{align}

Bu, silindirik koordinatlarda konum vektörünün beklenen ifadesidir.

\section*{Vektör Alanlarının Silindirik Koordinatlara Dönüşümü}

Verilen vektör alanları:
\begin{align}
    U_1 &= 2xy^2 + x^2 \\
    U_2 &= xy + z \\
    U_3 &= x^2y + z^2y + z
\end{align}

Bu alanları silindirik koordinatlara dönüştürmek için, Kartezyen değişkenleri silindirik koordinatlar cinsinden yazalım:

$x = \rho\cos\theta$, $y = \rho\sin\theta$, $z = z$

\subsection*{$U_1$ Vektör Alanının Dönüşümü}
\begin{align}
    U_1 &= 2xy^2 + x^2 \\
    &= 2(\rho\cos\theta)(\rho\sin\theta)^2 + (\rho\cos\theta)^2 \\
    &= 2\rho^3\cos\theta\sin^2\theta + \rho^2\cos^2\theta \\
    &= 2\rho^3\cos\theta\sin^2\theta + \rho^2\cos^2\theta
\end{align}

\subsection*{$U_2$ Vektör Alanının Dönüşümü}
\begin{align}
    U_2 &= xy + z \\
    &= (\rho\cos\theta)(\rho\sin\theta) + z \\
    &= \rho^2\cos\theta\sin\theta + z
\end{align}

\subsection*{$U_3$ Vektör Alanının Dönüşümü}
\begin{align}
    U_3 &= x^2y + z^2y + z \\
    &= (\rho\cos\theta)^2(\rho\sin\theta) + z^2(\rho\sin\theta) + z \\
    &= \rho^3\cos^2\theta\sin\theta + \rho z^2\sin\theta + z \\
    &= \rho^3\cos^2\theta\sin\theta + \rho z^2\sin\theta + z
\end{align}

\section*{Ödev: Küresel Koordinatlara Uygulama}

Küresel koordinatlarda benzer dönüşüm işlemlerini uygulayın. Küresel koordinat sistemi $(r, \theta, \phi)$ ile tanımlanır, burada:
\begin{itemize}
    \item $r$: Orijinden olan uzaklık
    \item $\theta$: $xy$-düzlemindeki açı (enlem açısı)
    \item $\phi$: $z$-ekseni ile yapılan açı (boylam açısı)
\end{itemize}

Küresel ve Kartezyen koordinatlar arasındaki dönüşüm ilişkileri:
\begin{align}
    x &= r\sin\phi\cos\theta \\
    y &= r\sin\phi\sin\theta \\
    z &= r\cos\phi
\end{align}

Tersi:
\begin{align}
    r &= \sqrt{x^2 + y^2 + z^2} \\
    \theta &= \arctan\left(\frac{y}{x}\right) \\
    \phi &= \arccos\left(\frac{z}{\sqrt{x^2 + y^2 + z^2}}\right)
\end{align}

Ödevinizde, silindirik koordinatlar için yaptığımız işlemlerin benzerini küresel koordinatlar için de yapmanız istenmektedir. Buna şunlar dahildir:
\begin{enumerate}
    \item Küresel koordinatlardaki birim vektörlerin Kartezyen birim vektörleri cinsinden ifadesi
    \item Dönüşüm matrisinin oluşturulması
    \item Konum vektörünün küresel koordinatlarda ifade edilmesi
    \item Verilen vektör alanlarının küresel koordinatlara dönüştürülmesi
\end{enumerate}

\end{document}