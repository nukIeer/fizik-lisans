\documentclass[11pt,letterpaper,twocolumn]{fenbil}
\usepackage{amsmath}
\usepackage{amssymb}
\usepackage{graphicx}
\usepackage{tikz}
\title{FMM - Fizikte Matematiksel Metotlar - Ders Notları (Devam)}
\author{Celal Ekrem Torun}
\date{6 Mart 2025}

\begin{document}
\twocolumn[\begin{@twocolumnfalse}

\begin{minipage}{0.15\textwidth}{
    }
\end{minipage}
\hspace{25pt}
\begin{minipage}{0.75\textwidth}
\vspace{5mm}
\Large{\textbf{FMM - Fizikte Matematiksel Metotlar - Ders 2 (6 Mart 2025)}}
    \vspace{3mm}
    
    \large{\textbf{Hazırlayan}; Celal Ekrem Torun}
    \vspace{2mm}
    
    \fontsize{0.35cm}{0.5cm}\selectfont \textit{Fizik Bölümü, İstanbul Üniversitesi\newline 
    Beyazıt, Fatih, İstanbul, Türkiye}
    
\end{minipage}

\small

\end{@twocolumnfalse}]

\section{Eğrisel Koordinat Sisteminin Taban Vektörleri ve Ölçek Çarpanları (Devam)}

Bir önceki bölümde, eğrisel koordinat sistemlerinin temel kavramlarını incelemiştik. Bu bölümde, taban vektörleri ve ölçek çarpanları konusuna devam edeceğiz.

\subsection{Teğet Vektörleri}

Eğrisel koordinat sisteminde, teğet vektörleri, her bir koordinatın değişim yönünü gösterir. Konum vektörü $\vec{r}(u_1, u_2, u_3)$ olmak üzere, teğet vektörleri şu şekilde tanımlanır:

\begin{equation}
\vec{t_1} = \frac{\partial \vec{r}}{\partial u_1}
\end{equation}
\begin{equation}
\vec{t_2} = \frac{\partial \vec{r}}{\partial u_2}
\end{equation}
\begin{equation}
\vec{t_3} = \frac{\partial \vec{r}}{\partial u_3}
\end{equation}

Bu vektörler, eğrisel koordinat sisteminin tabanını oluşturur.

\subsection{Taban Vektörleri}

Taban vektörleri, teğet vektörlerinin normalleştirilmesiyle elde edilir. Bu vektörler, birim uzunluğa sahiptir ve koordinat sisteminin yönünü gösterir. Taban vektörleri şu şekilde tanımlanır:

\begin{equation}
\hat{e_1} = \frac{\vec{t_1}}{|\vec{t_1}|} = \frac{1}{h_1} \vec{t_1}
\end{equation}
\begin{equation}
\hat{e_2} = \frac{\vec{t_2}}{|\vec{t_2}|} = \frac{1}{h_2} \vec{t_2}
\end{equation}
\begin{equation}
\hat{e_3} = \frac{\vec{t_3}}{|\vec{t_3}|} = \frac{1}{h_3} \vec{t_3}
\end{equation}

Burada $h_1$, $h_2$ ve $h_3$, ölçek çarpanlarıdır ve teğet vektörlerinin büyüklüklerine eşittir:

\begin{itemize}
    \item $h_1 = |\vec{t_1}| = \left| \frac{\partial \vec{r}}{\partial u_1} \right|$
    \item $h_2 = |\vec{t_2}| = \left| \frac{\partial \vec{r}}{\partial u_2} \right|$
    \item $h_3 = |\vec{t_3}| = \left| \frac{\partial \vec{r}}{\partial u_3} \right|$
\end{itemize}

\subsection{Geometrik Gösterim}

Aşağıdaki şema, taban vektörlerini ve eğrisel koordinatları küçük bir Kartezyen sisteminde göstermektedir:

\begin{figure}[htbp]
    \centering
    \begin{tikzpicture}
        % Eksenleri çiz
        \draw[thick, ->] (-3,0) -- (3,0) node[anchor=north west] {x};
        \draw[thick, ->] (0,-3) -- (0,3) node[anchor=south east] {y};
        \draw[thick, ->] (0,0) -- (1.5,1.5) node[anchor=west] {z};

        % Taban vektörleri
        \draw[blue, thick, ->] (0,0) -- (2,0) node[midway, below] {$\hat{e_1}$};
        \draw[red, thick, ->] (0,0) -- (0,2) node[midway, left] {$\hat{e_2}$};
        \draw[green, thick, ->] (0,0) -- (1,1) node[midway, above right] {$\hat{e_3}$};

        % Eğrisel yüzeyler
        \draw[dashed, blue] (2,0) arc (0:45:2);
        \draw[dashed, red] (0,2) arc (90:45:2);
    \end{tikzpicture}
    \caption{Taban Vektörleri ve Eğrisel Koordinatlar}
    \label{fig:taban_vektorleri}
\end{figure}

Bu şemada:

\begin{itemize}
    \item $\hat{e_1}$, $\hat{e_2}$ ve $\hat{e_3}$: Birim taban vektörlerini temsil eder.
    \item Eğriler: Eğrisel koordinatların değişimini temsil eder.
    \item $P(u_1, u_2, u_3)$: Eğrisel koordinat sistemindeki bir noktayı temsil eder.
\end{itemize}

\subsection{Konum Vektörünün İfadesi}

Konum vektörü $\vec{r}$, Kartezyen koordinatlarda şu şekilde ifade edilir:

\begin{equation}
\vec{r} = x \, \hat{e_x} + y \, \hat{e_y} + z \, \hat{e_z}
\end{equation}

Eğrisel koordinatlarda ise, teğet vektörleri kullanılarak ifade edilebilir:

\begin{equation}
\vec{t_1} = \frac{\partial \vec{r}}{\partial u_1} = \frac{\partial x}{\partial u_1} \hat{e_x} + \frac{\partial y}{\partial u_1} \hat{e_y} + \frac{\partial z}{\partial u_1} \hat{e_z}
\end{equation}

Bu ifade, $u_1$ koordinatındaki değişimin, konum vektörünü nasıl etkilediğini gösterir.

\end{document}
